\documentclass[10pt,DIV=15,a4paper]{scrartcl}
\usepackage[utf8]{inputenc}
\usepackage[ngerman]{babel}
\usepackage[T1]{fontenc}
\usepackage{lmodern}


\usepackage{amsmath}
\usepackage{amsfonts}
\usepackage{amssymb}

\usepackage{xcolor}
\usepackage{tikz}
\usepackage{pgfplots}
\pgfplotsset{compat=1.18}
\usepackage{pgfplotstable}
\usetikzlibrary{plotmarks}

\pgfplotsset{
    discard if not/.style 2 args={
        x filter/.code={
            \edef\tempa{\thisrow{#1}}
            \edef\tempb{#2}
            \ifx\tempa\tempb
            \else
                \def\pgfmathresult{inf}
            \fi
        }
    }
}

\pgfplotscreateplotcyclelist{mylist}{
  {blue,mark=x},
  {olive,mark=+},
  {red,mark=o},
  {brown,mark=diamond},
  {violet,mark=triangle}% <-- don't add a comma here
}

\begin{document}

  \newcommand{\datatablename}{performance.csv}
  %\pgfplotstableread[col sep=comma]{performance.csv}\datatable
  \newcommand{\performanceplot}[2]{
    \addplot+ [
      sharp plot,
      discard if not={dichte}{#1}
    ] table[x=groesse,y=#2,col sep=comma] {\datatablename};
  }

  \newcommand{\plottingDichte}[2][]{
    \begin{tikzpicture}
      \begin{axis}[
        xlabel={Grösse der Matrix},
        ylabel=Zeit,
       % align right:
       legend style={
       cells={anchor=east},
       legend pos=outer north east,
       },
       cycle list name=mylist
       ]

        \performanceplot{#2}{#1Zeit1}
        \performanceplot{#2}{#1Zeit2}
        \performanceplot{#2}{#1Zeit3}
        \performanceplot{#2}{#1Zeit4}
        \legend{\textsc{Laplace}, \textsc{Laplace} mit Absuchen, Dreiecksform, Dreieckf. mit Pivotisierung}
      \end{axis}
    \end{tikzpicture}
  }

  \newcommand{\DichtePlots}[1]{
    \begin{figure}[h]
    \centering
      \plottingDichte{#1}
    \caption{Durchschnittliche Laufzeit der unterschiedlichen Algorithmen bei $#1\%$ Besetzungsgrad.}
    \end{figure}

    \begin{figure}[h]
    \centering
      \plottingDichte[min]{#1}
    \caption{Kleinste Laufzeit der unterschiedlichen Algorithmen bei $#1\%$ Besetzungsgrad.}
    \end{figure}

    \begin{figure}[h]
    \centering
      \plottingDichte[max]{#1}
    \caption{Größte der unterschiedlichen Algorithmen bei $#1\%$ Besetzungsgrad.}
    \end{figure}
  }

  \section{Vergleich der Komplexität der Algorithmen bei gleichem Besetzungsgrad}



  \DichtePlots{20}
  \DichtePlots{40}
  \DichtePlots{60}
  \DichtePlots{80}
  \DichtePlots{100}

  \clearpage

  \section{Vergleich der Komplexität des einzelnen Algorithmus bei unterschiedlichen Besetzungsgraden}

  \begin{figure}[h]
  \centering
    \begin{tikzpicture}
      \begin{axis}[
        xlabel={Grösse der Matrix},
        ylabel=Zeit,
       % align right:
       legend style={
       cells={anchor=east},
       legend pos=outer north east,
       }
       ]

      \performanceplot{20} {Zeit1}
      \performanceplot{40} {Zeit1}
      \performanceplot{60} {Zeit1}
      \performanceplot{80} {Zeit1}
      \performanceplot{100}{Zeit1}

      \legend{$20\%$,$40\%$,$60\%$,$80\%$,$100\%$}

      \end{axis}
    \end{tikzpicture}
  \caption{Laufzeitverhalten des naiven \textsc{Laplace}schen Entwicklungssatzes bei unterschiedlichen Besetzungsgraden}
  \end{figure}

  \begin{figure}[h]
  \centering
    \begin{tikzpicture}
      \begin{axis}[
        xlabel={Grösse der Matrix},
        ylabel=Zeit,
       % align right:
       legend style={
       cells={anchor=east},
       legend pos=outer north east,
       }
       ]

      \performanceplot{20} {Zeit2}
      \performanceplot{40} {Zeit2}
      \performanceplot{60} {Zeit2}
      \performanceplot{80} {Zeit2}
      \performanceplot{100}{Zeit2}

      \legend{$20\%$,$40\%$,$60\%$,$80\%$,$100\%$}

      \end{axis}
    \end{tikzpicture}
  \caption{Laufzeitverhalten des \textsc{Laplace}schen Entwicklungssatzes mit Absuchen bei unterschiedlichen Besetzungsgraden}
  \end{figure}

  \begin{figure}[h]
  \centering
    \begin{tikzpicture}
      \begin{axis}[
        xlabel={Grösse der Matrix},
        ylabel=Zeit,
       % align right:
       legend style={
       cells={anchor=east},
       legend pos=outer north east,
       }
       ]

      \performanceplot{20} {Zeit3}
      \performanceplot{40} {Zeit3}
      \performanceplot{60} {Zeit3}
      \performanceplot{80} {Zeit3}
      \performanceplot{100}{Zeit3}

      \legend{$20\%$,$40\%$,$60\%$,$80\%$,$100\%$}

      \end{axis}
    \end{tikzpicture}
  \caption{Laufzeitverhalten des Überführens in Dreiecksform bei unterschiedlichen Besetzungsgraden}
  \end{figure}

\end{document}
